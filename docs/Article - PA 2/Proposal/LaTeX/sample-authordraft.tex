%%
%% This is file `sample-manuscript.tex',
%% generated with the docstrip utility.
%%
%% The original source files were:
%%
%% samples.dtx  (with options: `manuscript')
%% 
%% IMPORTANT NOTICE:
%% 
%% For the copyright see the source file.
%% 
%% Any modified versions of this file must be renamed
%% with new filenames distinct from sample-manuscript.tex.
%% 
%% For distribution of the original source see the terms
%% for copying and modification in the file samples.dtx.
%% 
%% This generated file may be distributed as long as the
%% original source files, as listed above, are part of the
%% same distribution. (The sources need not necessarily be
%% in the same archive or directory.)
%%
%% Commands for TeXCount
%TC:macro \cite [option:text,text]
%TC:macro \citep [option:text,text]
%TC:macro \citet [option:text,text]
%TC:envir table 0 1
%TC:envir table* 0 1
%TC:envir tabular [ignore] word
%TC:envir displaymath 0 word
%TC:envir math 0 word
%TC:envir comment 0 0
%%
%%
%% The first command in your LaTeX source must be the \documentclass command.
%%%% Small single column format, used for CIE, CSUR, DTRAP, JACM, JDIQ, JEA, JERIC, JETC, PACMCGIT, TAAS, TACCESS, TACO, TALG, TALLIP (formerly TALIP), TCPS, TDSCI, TEAC, TECS, TELO, THRI, TIIS, TIOT, TISSEC, TIST, TKDD, TMIS, TOCE, TOCHI, TOCL, TOCS, TOCT, TODAES, TODS, TOIS, TOIT, TOMACS, TOMM (formerly TOMCCAP), TOMPECS, TOMS, TOPC, TOPLAS, TOPS, TOS, TOSEM, TOSN, TQC, TRETS, TSAS, TSC, TSLP, TWEB.
% \documentclass[acmsmall]{acmart}

%%%% Large single column format, used for IMWUT, JOCCH, PACMPL, POMACS, TAP, PACMHCI
% \documentclass[acmlarge,screen]{acmart}

%%%% Large double column format, used for TOG
% \documentclass[acmtog, authorversion]{acmart}

%%%% Generic manuscript mode, required for submission
%%%% and peer review
\documentclass[manuscript,screen,review]{acmart}
\usepackage[brazil]{babel}
\usepackage[utf8]{inputenc}
%% Fonts used in the template cannot be substituted; margin 
%% adjustments are not allowed.
%%
%% \BibTeX command to typeset BibTeX logo in the docs
\AtBeginDocument{%
  \providecommand\BibTeX{{%
    \normalfont B\kern-0.5em{\scshape i\kern-0.25em b}\kern-0.8em\TeX}}}

%% Rights management information.  This information is sent to you
%% when you complete the rights form.  These commands have SAMPLE
%% values in them; it is your responsibility as an author to replace
%% the commands and values with those provided to you when you
%% complete the rights form.

\setcopyright{acmcopyright}
\copyrightyear{2023}
\acmYear{2023}
\acmDOI{XXXXXXX.XXXXXXX}

%% These commands are for a PROCEEDINGS abstract or paper.
\acmConference[UFSJ]{Make sure to enter the correct
  conference title from your rights confirmation emai}{Maio 09-05,
  2023}{São João del-Rei, MG}
%
%  Uncomment \acmBooktitle if th title of the proceedings is different
%  from ``Proceedings of ...''!
%

%%
%% Submission ID.
%% Use this when submitting an article to a sponsored event. You'll
%% receive a unique submission ID from the organizers
%% of the event, and this ID should be used as the parameter to this command.
%%\acmSubmissionID{123-A56-BU3}

%%
%% For managing citations, it is recommended to use bibliography
%% files in BibTeX format.
%%
%% You can then either use BibTeX with the ACM-Reference-Format style,
%% or BibLaTeX with the acmnumeric or acmauthoryear sytles, that include
%% support for advanced citation of software artefact from the
%% biblatex-software package, also separately available on CTAN.
%%
%% Look at the sample-*-biblatex.tex files for templates showcasing
%% the biblatex styles.
%%

%%
%% The majority of ACM publications use numbered citations and
%% references.  The command \citestyle{authoryear} switches to the
%% "author year" style.
%%
%% If you are preparing content for an event
%% sponsored by ACM SIGGRAPH, you must use the "author year" style of
%% citations and references.
%% Uncommenting
%% the next command will enable that style.
%%\citestyle{acmauthoryear}
\settopmatter{printacmref=false}
%%
%% end of the preamble, start of the body of the document source.
\begin{document}

%%
%% The "title" command has an optional parameter,
%% allowing the author to define a "short title" to be used in page headers.
\title{Detecção de URLs Maliciosas}

%%
%% The "author" command and its associated commands are used to define
%% the authors and their affiliations.
%% Of note is the shared affiliation of the first two authors, and the
%% "authornote" and "authornotemark" commands
%% used to denote shared contribution to the research.
\author{Julio Rodrigues}
\email{julio.csr.271@aluno.ufsj.edu.br}
\affiliation{%
  \institution{Universidade Federal de São João del-Rei}
  \city{São João del-Rei}
  \state{Minas Gerais}
  \country{Brasil}
  \postcode{36.301-360}
}

%%
%% By default, the full list of authors will be used in the page
%% headers. Often, this list is too long, and will overlap
%% other information printed in the page headers. This command allows
%% the author to define a more concise list
%% of authors' names for this purpose.
\renewcommand{\shortauthors}{Rodrigues, J.}

%%
%% The abstract is a short summary of the work to be presented in the
%% article.

%%
%% Keywords. The author(s) should pick words that accurately describe
%% the work being presented. Separate the keywords with commas.

%% A "teaser" image appears between the author and affiliation
%% information and the body of the document, and typically spans the
%% page.

\received{09 Maio 2023}

%%
%% This command processes the author and affiliation and title
%% information and builds the first part of the formatted document.
\maketitle

\section{Introdução}

De modo geral, o elo mais fraco na cibersegurança é o ser humano, mais especificamente o usuário. URLs são vias rápidas e diretas para possibilitar ações maliciosas na internet, bastando no máximo alguns cliques para efetivar um ataque. Por isto, o potencial de infecção por este meio é exponencial, embora a efetividade desses ataques dependa de características adicionais da URL e das técnicas de camuflagem utilizadas.

Em 2019, o Brasil estava no Top 15 entre os países com o maior número de vítimas de ataques com URLs maliciosas \cite{evilurl}. Atualmente, estas estatísticas possivelmente podem apresentar-se bem piores, devido ao período transcorrido da pandemia de \emph{Covid-19}, com a elevação no uso da tecnologia de modo geral.  É um campo em que existe uma constante evolução dos dois lados conflitantes. O lado de combate, aprimorando as técnicas para detecção, e o lado de ataque, evoluindo as técnicas de camuflagem das URLs.

No caso do \emph{phishing}, por exemplo, principalmente nos casos de ataques por email, geralmente a intenção é sempre provocar alguma urgência no assunto, instigando o acesso da vítima à página sem nenhuma checagem. Por isto, são utilizadas técnicas de engenharia social na tentativa de assemelhar ao máximo URLs maliciosas com URLs seguras. É basicamente uma verdadeira guerra entre os lados, alternando entre invenção de novas técnicas e detecção das novas técnicas, isto tudo na esperança de um dia não existirem mais opções viáveis para formular ataques deste tipo.

\section{Motivação}

Em relação à motivação para execução deste trabalho, até o momento existe uma pergunta central a qual deseja-se responder:

\begin{itemize}
    \item RQ1: \textit{Quais são as principais características que definem a natureza de uma URL?} 
\end{itemize}

O intuito de trabalhar sobre esta questão única é entender:

\begin{enumerate}
    \item \textit{"Em que aspectos cada URL se destaca?"};
    \item \textit{"Que informações são suficientes para distinção entre URLs seguras e maliciosas?"};
    \item \textit{"O que separa \textbf{phishing} de \textbf{malware}?"}.
\end{enumerate}

Estas são as questões as quais se dará maior atenção no desenvolvimento deste trabalho, e as quais se espera responder, cumprindo ao menos os objetivos iniciais definidos.

\pagebreak

\section{Cronograma}

Nesta seção, serão citados os trabalhos relacionados que servirão como base inicial para os primeiros passos no desenvolvimento deste trabalho. Também será apresentado um cronograma de desenvolvimento provisório, o qual possivelmente poderá sofrer alterações diversas, dependendo das direções seguidas (ou eventuais atrasos) ao longo da produção do trabalho.

\subsection{Trabalhos Relacionados}

Inicialmente, foram selecionados três artigos para auxiliar no desenvolvimento. Dois deles \cite{10.1145/3548606.3560615} \cite{10.1145/3465481.3470029} estão diretamente relacionados com o problema à ser tratado neste trabalho: detecção de URLs maliciosas. O segundo \cite{10.5555/3455716.3455827} está relacionado à uma abordagem que se pretende avaliar em estágios mais avançados do desenvolvimento: extração de \emph{meta-features}.

\subsection{Desenvolvimento}

Para primeira apresentação parcial, pretende-se agregar mais bases de dados e continuar a análise exploratória de forma mais extensa. Já no intervalo entre as entregas parciais, os esforços seriam concentrados na criação de novos atributos. No entanto, o foco não seria na parte léxica da URL, e sim, no conteúdo, tentando extrair informação útil do código HTML da página, ou até mesmo de dados relacionados à rede. O objetivo neste ponto, é encontrar atributos relevantes que possam definir de forma mais clara as classes.

Para a parcial 2, cessada a etapa de construção dos atributos, iniciaria-se o processo de análise e seleção dos mesmos, identificando a relevância de cada um no(s) modelo(s). Em seguida, seriam selecionados um conjunto de algoritmos de \emph{machine learning} para executar os treinamentos iniciais, e realizar um comparativo com outros trabalhos da literatura, destacando similaridades, limitações, pontos positivos e negativos.

Por fim, a última etapa seria dedicada para a exploração da extração de \emph{meta-features}, processo o qual ainda se faz um tanto quanto nebuloso no estágio de proposta deste trabalho. No entanto, indenpendentemente dos avanços e resultados obtidos com as \emph{meta-features}, nesta etapa se dará a construção do modelo final, para então validar os resultados e finalizar a escrita do artigo, concluindo o trabalho. Todas as etapas de desevolvimento estão sumarizadas na Tabela \ref{tab:exampleTab1}.

\begin{table}[H]
    \centering
    \caption{Cronograma de Desenvolvimento}
    \label{tab:exampleTab1}
    \begin{tabular}{c|c|c}
        \hline
        \textbf{Data} & \textbf{Atividade} & \textbf{Desenvolvimento}\\
        \hline
        25/05/2023 & Parcial I & Análise das Bases\\
        06/06/2023 & --- & Criação de Atributos\\
        13/06/2023 & Parcial II & Comparativo com Trabalhos\\
        15/06/2023 & --- & \emph{Meta-features}\\
        27/06/2023 & Final & Modelo Final e Artigo\\
        \hline
    \end{tabular}
\end{table}

%%
%% The next two lines define the bibliography style to be used, and
%% the bibliography file.
\bibliographystyle{ACM-Reference-Format}
\bibliography{sample-base}


\end{document}
\endinput
%%
%% End of file `sample-authordraft.tex'.
