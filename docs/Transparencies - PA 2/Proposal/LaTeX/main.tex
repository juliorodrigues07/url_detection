\documentclass{beamer}
\usepackage{hyperref}
\usepackage{textcomp}
\usepackage[CJKmath=true, AutoFakeBold = true]{xeCJK}
% \usepackage[T1]{fontenc}
\setCJKmainfont{AR PL KaitiM GB}
\usepackage{latexsym,xcolor,multicol,booktabs,calligra}
\usepackage{amssymb,amsfonts,amsmath,amsthm,mathrsfs,mathptmx}
\usepackage{graphicx,pstricks,listings,stackengine}
\usefonttheme[onlymath]{serif}
\usepackage[brazil]{babel}

\renewcommand{\today}{4 de Maio de 2023}
\renewcommand{\alert}[1]{\textbf{\color{swufe}#1}}

\author[Julio Rodrigues (UFSJ)]{Julio Cesar da Silva Rodrigues\inst{1}}
\title[Proposta - TP2]{Detecção de URLs Maliciosas}
\subtitle{Mineração de Dados Aplicada}
\institute[UFSJ]
{
    \inst{1} 
    Universidade Federal de São João del-Rei \\
    Curso de Ciência da Computação \\
    \textit{julio.csr.271@aluno.ufsj.edu.br}\\
}

\usetheme{Warsaw}
\setbeamertemplate{page number in head/foot}[totalframenumber]
% \usepackage{SWUFE}

\def\cmd#1{\texttt{\color[RGB]{0, 0, 139}\footnotesize $\backslash$#1}}
\def\env#1{\texttt{\color[RGB]{0, 0, 139}\footnotesize #1}}

\lstset{
    language=[LaTeX]TeX,
    basicstyle=\ttfamily\footnotesize,
    keywordstyle=\bfseries\color[RGB]{0, 0, 139},
    stringstyle=\color[RGB]{50, 50, 50},
    numbers=left,
    numberstyle=\small\color{gray},
    rulesepcolor=\color{red!20!green!20!blue!20},
    frame=shadowbox,
}

\begin{document}

\begin{frame}[plain]
    \titlepage
    \vspace*{-2cm}
    \begin{figure}[htpb]
        \begin{center}
            \includegraphics[width=0.4\linewidth]{pic/LogoUFSJ.PNG}
        \end{center}
    \end{figure}
    \begin{center}
        \footnotesize 4 de Maio de 2023
    \end{center}
\end{frame}

\begin{frame}{Conteúdo}
    \tableofcontents[sectionstyle=show,subsectionstyle=show/shaded/hide,subsubsectionstyle=show/shaded/hide]
\end{frame}

\section{Contextualização}

\begin{frame}{Conteúdo} 
     \tableofcontents[currentsection]
\end{frame}

\begin{frame}{URLs Maliciosas}

    \begin{itemize}[<+-| alert@+>]
        \setlength{\itemsep}{10pt}
        \item Via rápida e direta para aplicar crimes cibernéticos;
        \item Potencial de infecção exponencial;
        \item Brasil no Top 15 com maior número de vítimas \cite{evilurl};
        \item Evolução nas técnicas de camuflagem e detecção.
    \end{itemize}
    
\end{frame}

\section{Motivação}

\begin{frame}{Conteúdo} 
     \tableofcontents[currentsection]
\end{frame}

\begin{frame}{Perguntas e Objetivos}
    
    \begin{itemize}[<+-| alert@+>]
        \setlength{\itemsep}{10pt}
        \item Quais são as principais características que definem a natureza de uma URL?
        \item Existe vantagem na aplicação de \emph{instance selection} em balanceamento?
        \item É possível obter resultados equiparáveis com mais classes?
        \item Influência de algoritmos em resultados com a base final.
    \end{itemize}
    
\end{frame}

\section{Cronograma}

\begin{frame}{Conteúdo} 
     \tableofcontents[currentsection]
\end{frame}

\subsection{I. Artigos}

\begin{frame}{Base de Artigos Selecionados}

    \begin{enumerate}[<+-| alert@+>]
        \setlength{\itemsep}{10pt}
        \item Detecting Malicious URLs Using Machine Learning Techniques: Review and Research Directions (IEEE \emph{Xplore\textregistered});
        \item A Survey of Intelligent Detection Designs of HTML URL Phishing Attacks (IEEE \emph{Xplore\textregistered});
        \item A Comparative Survey of Instance Selection Methods applied to NonNeural and Transformer-Based Text Classification (ACM Digital Library);
        \item An Assessment of Lexical, Network, and Content-Based Features for Detecting Malicious URLs Using Machine Learning and Deep Learning Models (Hindawi). \nocite{*}
    \end{enumerate}
    
\end{frame}

\subsection{II Desenvolvimento}

\begin{frame}{Etapas}
    
    \begin{itemize}[<+-| alert@+>]
        \setlength{\itemsep}{10pt}
        \item Parcial I - 25/05/2023:
        \begin{enumerate}
            \item Análise exploratória aprofundada da base;
            \item Construção de novos atributos. 
        \end{enumerate}

        \item Parcial II - 13/06/2023:
        \begin{enumerate}
            \item Análise e seleção de atributos;
            \item Balanceamento da base de dados.
        \end{enumerate}

        \item Final - 27/06/2023:
        \begin{enumerate}
            \item Construção dos modelos de \emph{machine learning};
            \item Validação dos resultados e finalização da escrita do artigo.
        \end{enumerate}
    \end{itemize}
    
\end{frame}

\section*{Final}

\begin{frame}[allowframebreaks]{Referências}
    \scriptsize\bibliographystyle{plainnat}
    \bibliography{ref}
    % \bibliographystyle{alpha}
\end{frame}

\end{document}
