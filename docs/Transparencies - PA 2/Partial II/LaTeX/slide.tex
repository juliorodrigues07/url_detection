\documentclass{beamer}
\usepackage{hyperref}
\usepackage{textcomp}
\usepackage[CJKmath=true, AutoFakeBold = true]{xeCJK}
% \usepackage[T1]{fontenc}
\setCJKmainfont{AR PL KaitiM GB}
\usepackage{latexsym,xcolor,multicol,booktabs,calligra}
\usepackage{amssymb,amsfonts,amsmath,amsthm,mathrsfs,mathptmx}
\usepackage{graphicx,pstricks,listings,stackengine}
\usefonttheme[onlymath]{serif}
\usepackage[brazil]{babel}

\renewcommand{\today}{13 de Junho de 2023}
\renewcommand{\alert}[1]{\textbf{\color{swufe}#1}}

\author[Julio Rodrigues (UFSJ)]{Julio Cesar da Silva Rodrigues\inst{1}}
\title[Mineração de Dados Aplicada - TP 2 - Parcial II]{Detecção de URLs Maliciosas}
\subtitle{Mineração de Dados Aplicada}
\institute[UFSJ]
{
    \inst{1} 
    Universidade Federal de São João del-Rei \\
    Curso de Ciência da Computação \\
    \textit{julio.csr.271@aluno.ufsj.edu.br}\\
}

\usetheme{Warsaw}
\setbeamertemplate{page number in head/foot}[totalframenumber]
% \usepackage{SWUFE}

\def\cmd#1{\texttt{\color[RGB]{0, 0, 139}\footnotesize $\backslash$#1}}
\def\env#1{\texttt{\color[RGB]{0, 0, 139}\footnotesize #1}}

\lstset{
    language=[LaTeX]TeX,
    basicstyle=\ttfamily\footnotesize,
    keywordstyle=\bfseries\color[RGB]{0, 0, 139},
    stringstyle=\color[RGB]{50, 50, 50},
    numbers=left,
    numberstyle=\small\color{gray},
    rulesepcolor=\color{red!20!green!20!blue!20},
    frame=shadowbox,
}

\begin{document}

\begin{frame}[plain]
    \titlepage
    \vspace*{-1.8cm}
    \begin{figure}[htpb]
        \begin{center}
            \includegraphics[width=0.35\linewidth]{pic/LogoUFSJ.PNG}
        \end{center}
    \end{figure}
    \begin{center}
        \footnotesize 13 de Junho de 2023
    \end{center}
\end{frame}

\begin{frame}{Roteiro}
    \tableofcontents[sectionstyle=show,subsectionstyle=show/shaded/hide,subsubsectionstyle=show/shaded/hide]
\end{frame}

\section{Novo Atributo}

\begin{frame}{Roteiro} 
     \tableofcontents[currentsection]
\end{frame}

\begin{frame}{Status das Páginas}

    \begin{itemize}
        \setlength{\itemsep}{10pt}
        \item URLs maliciosas possuem curto tempo de vida;
        \item Código HTTP retornado pode ajudar a destacar URLs deste tipo;
        \item Impacto observado na classificação foi positivo;
        \item Redução nos falsos negativos (\emph{malware} classificado como seguro).
    \end{itemize}
    
\end{frame}

\begin{frame}{Holdout 80 | 20}

    \begin{block}{}
\begin{table}
\centering

\resizebox{\columnwidth}{!}{
\begin{tabular}{|c|c|c|c|c|}

%\begin{tabularx}{\columnwidth}{|lcccc|}
\hline 
 
\multicolumn{4}{|c|}{XGBoost}\\ 
\hline 
\rowcolor{}

Class & Precision & Recall & F1-Score\\
 \hline 
 %\rowcolor{Gray}
%\multicolumn{5}{|c|}{Dictionary} \\ 
% \hline 
Benign & 0.93 & 0.98 & 0.95\\   %\hdashline
Defacement & 0.94 & 0.97 & 0.95\\   %\hdashline
Phishing & 0.96 & 0.89 & 0.93\\   %\hdashline
Malware & 0.94 & 0.82 & 0.88\\   %\hdashline
\hline
\end{tabular}
}
%\end{tabularx}
%\label{se:Dic}
\end{table}
\end{block}
\note[item]{.}

\end{frame}

\section{Balanceamento}

\begin{frame}{Roteiro} 
     \tableofcontents[currentsection]
\end{frame}

\subsection{I. Benign}

\begin{frame}{Classe \emph{benign}}

    \begin{itemize}
        \setlength{\itemsep}{10pt}
        \item Correspondem à quase 70\% do total da base;
        \item Classes \emph{defacement} e \emph{phishing} em relação à \emph{benign} possuem:
        \begin{enumerate}
            \vspace{0.2cm}
            \setlength{\itemsep}{10pt}      
            \item Macros F1 próximas;
            \item Cada uma apresenta menos que 25\% da quantidade de instâncias.
        \end{enumerate} 
        \item \emph{Undersampling} aleatório;
        \item Remover grande parte destas instâncias, deixando alguma margem;
        \item Evitar perca de informação.
    \end{itemize}

\end{frame}

\subsection{II. Phishing}

\begin{frame}{Classe \emph{phishing}}

    \begin{itemize}
        \setlength{\itemsep}{10pt}
        \item \emph{PhishTank}\footnotemark para suprir o déficit de instâncias de \emph{phishing};
        \item \emph{Scraper} básico coletando URLs;
        \item Vantagem em relação à \emph{oversampling} aleatório.
    \end{itemize}

    \footnotetext[1]{\tiny Disponível em: \url{https://phishtank.org/phish_archive.php}}
    
\end{frame}

\subsection{III. Defacement e Malware}

\begin{frame}{Classes \emph{defacement} e \emph{malware}}

    \begin{itemize}
        \setlength{\itemsep}{10pt}
        \item Dificuldade em encontrar bases de dados com a quantidade de instâncias necessária;
        \item Utilização de \emph{SMOTE} \cite{DBLP:journals/corr/abs-1106-1813} em vez de \emph{oversampling} aleatório;
        \item Gerar novas instâncias com base em dados existentes;
    \end{itemize}

\end{frame}

\subsection{IV. Base Final}

\begin{frame}{Base de Dados Final}

    \begin{itemize}
        \setlength{\itemsep}{10pt}
        \item 800 mil instâncias, totalmente balanceada;
        \item 8 atributos e 1 classe (4 valores distintos);
        \item Composta por URLs de bases de dados do \emph{Kaggle}\footnotemark e \emph{PhishTank}.
    \end{itemize}

    \footnotetext[2]{\tiny Disponível em: \url{https://www.kaggle.com/datasets/sid321axn/malicious-urls-dataset}}
    
\end{frame}

\section{Resultados}

\begin{frame}{Roteiro} 
     \tableofcontents[currentsection]
\end{frame}

\begin{frame}{Comparativo de Parciais}

    \begin{block}{}
\begin{table}
\centering

\resizebox{\columnwidth}{!}{
\begin{tabular}{|c|c|c|}

%\begin{tabularx}{\columnwidth}{|lcccc|}
\hline 
 
\multicolumn{3}{|c|}{TP 2 - Parcial I}\\ 
\hline 

 Algoritmo & Média  & Desvio Padrão\\
 \hline 
 %\rowcolor{Gray}
%\multicolumn{5}{|c|}{Dictionary} \\ 
% \hline 
Regressão Logística & 0,5780295368328738 & 0,0021293237223429956\\   %\hdashline
XGBoost & 0,8568910936179079 & 0,0017065729226258411\\  %\hdashline  
  \hline 


\end{tabular}
}
%\end{tabularx}
%\label{se:Dic}
\end{table}
\end{block}
\note[item]{.}


\begin{block}{}
\begin{table}
\centering

\resizebox{\columnwidth}{!}{
\begin{tabular}{|c|c|c|}

%\begin{tabularx}{\columnwidth}{|lcccc|}
\hline 
 
\multicolumn{3}{|c|}{TP 2 - Parcial II}\\ 
\hline 

 Algoritmo & Média  & Desvio Padrão\\
 \hline 
 %\rowcolor{Gray}
%\multicolumn{5}{|c|}{Dictionary} \\ 
% \hline 
Regressão Logística & 0,67668635532121 & 0,0015789916679593162\\   %\hdashline
XGBoost & 0,9324923443237593 & 0,0008784997591179979\\  %\hdashline 
  \hline 


\end{tabular}
}
%\end{tabularx}
%\label{se:Dic}
\end{table}
\end{block}
\note[item]{.}
    
\end{frame}

\begin{frame}{Teste t de dupla cauda}

    \begin{itemize}
        \setlength{\itemsep}{10pt}
        \item Valores:
        \begin{enumerate}
            \setlength{\itemsep}{10pt}
            \item \(\alpha = 0,05\);
            \item \(t = 177,42420397526064\);
            \item \(p = 1,0791746370249748 \times 10^{-10}\).
        \end{enumerate}
        \item Hipótese nula rejeitada;
        \item Modelos estatisticamente distintos;
        \item XGBoost com nítida superioridade.
    \end{itemize}
    
\end{frame}

\section{Conclusão}

\begin{frame}{Roteiro} 
     \tableofcontents[currentsection]
\end{frame}

\begin{frame}{Faltou...}

    \begin{itemize}
        \setlength{\itemsep}{10pt}
        \item Selecionar novos algoritmos para teste;
        \item Comparação com trabalhos relacionados;
        \item Busca por novos atributos?.
    \end{itemize}
    
\end{frame}

\begin{frame}{Referências}
    \scriptsize\bibliographystyle{apalike}
    \bibliography{ref}
\end{frame}

\end{document}